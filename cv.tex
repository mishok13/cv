\documentclass[margin]{res}

\usepackage{helvetica}
\textwidth=5.2in
\usepackage{hyperref}
\hypersetup{colorlinks}
\usepackage{color}
\definecolor{darkred}{rgb}{0.4,0,0}
\definecolor{darkgreen}{rgb}{0,0.5,0}
\definecolor{darkblue}{rgb}{0,0,0.5}
\hypersetup{
  colorlinks,
  linkcolor=darkblue,
  filecolor=darkgreen,
  urlcolor=darkred,
  citecolor=darkblue}

\begin{document}

\name{Andrii V. Mishkovskyi\\[12pt]} % the \\[12pt] adds a blank line after name

\address{{\bf Address} \\ Mercatorplein 36-K \\ 1056CL, Amsterdam, The Netherlands}

\address{{\bf Contacts} \\ +31 (62) 980-61-41 \\ contact@mishkovskyi.net \\ skype: andriy.mishkovskyy \\ \href{http://mishkovskyi.net}{mishkovskyi.net}}

\begin{resume}

% \section{Objective}
% To work on a product which solves real problems, preferrably using reliable and suit-
% able tools. I d love to work in environment where I can not only contribute, but also
% learn from the team and the products developed.

\section{Education}
Bachelor of Computer Science, National University of Ukraine ``Kyiv Politechnical Institute'' \\
2003 -- 2007

\section{Experience}
{\bf Software Engineer,} CloudMade \hfill August 2008 -- present

CloudMade is GIS company that uses free
(``free'' as in ``libre'') data from OpenStreetMap project as
the base for its services. I'm responsible for tile and
vector servers development. Authorization servers are also
being supported by me.\\

\begin{itemize} \itemsep -1pt
\item Designed and implemented latest version of the tile
  server. Tile server itself supports
  \href{http://developers.cloudmade.com/projects/show/tiles}{tile} and
  \href{http://developers.cloudmade.com/projects/show/static-maps}{staticmaps}
  services.
  Tile server uses smart queue messaging control
  to improve stability and enhance scalability. Cache management system
  described later also improves cache hit ratio and
  allows to update tiles seamlessly. Tile server uses
  Twisted for networking, queue control and infrastructure messaging
  and Mapnik for rendering.
  The previous version was an ad hoc mix of C, C++ and Python.
\item Designed and implemented cache management system, which is
  responsible for updating tiles and staticmaps images in the
  background, thus minimizing serve time and overall load on
  services. The software is written in Python and uses RabbitMQ
  (with another backend supporting BDB) for the update queue and
  SimpleDB (interchangable with PostgreSQL or BDB) for storing
  meta-information about each image re-rendered.
\item Designed and implemented vector maps service.
  \href{http://developers.cloudmade.com/projects/show/vector-stream-server}{Vectormaps}
  service is a very important part of
  navigation pack CloudMade provides and is used by many
  mobile applications.
  I'm the principal author of all APIs this service
  provides. The backend itself is a mix of Pylons
  and GeoAlchemy (SQLAlchemy for GIS databases).
\item Developed Python API
  \href{http://developers.cloudmade.com/projects/show/python-lib}{library}
  from scratch,
  which is  used by developers of Java and Ruby
  libraries as a model to follow. The API is basically a wrap over
  REST-like HTTP APIs CloudMade provides. The library features simple
  query builder.
\item Open-source involvement -- Mapnik
  (PostGIS plugin, SRTM plugin,
  Python bindings), occasional patches to CPython.
\end{itemize}

{\bf Software Engineer,} Visonic Ltd. \hfill July 2007 -- August 2008

Visonic is an Israel-based manufacturer of security systems.
I was responsible for supporting legacy desktop software and
also got involved into development of several web-based
products.\\

\begin{itemize} \itemsep -1pt
\item Support of 10-year old new MFC application
  that was used to program embedded devices. The application itself
  has been supporting different kinds of devices, oldest of which
  were almost 20 years old. In the core of the application sits
  sophisticated mainloop, which makes interaction with serial port
  programmable devices as flawless as it is possible.




















  As the application doesn't use threads, the callback style is
  used to program actual functionality behind the interface.
\item Wrote lots of Python and XSLT scripts
  crunching XML files. Most of these scripts had curses interface +
  the usual optparse one. One of the hardest things about this was
  the need to support at least Python 2.2, because most of these scripts
  were to be run on old Red Hat installations.
\item Created in-house wrapper of
  \href{http://www.hyperrealm.com/libconfig/}{libconfig} using SWIG for initial
  implementation and later rewriting it completely using only Python C API, to make
  the interface less cumbersome. The wrapping itself was never released to
  opensource, because of management direct objection.
\end{itemize}



{\bf Tier 1/2 Technical Support Engineer,} Telesystems of Ukraine \hfill February 2006 -- July 2007

% Tier 1 support
% Worked as tech. support engineer for local ADSL internet
% service provider.\\

% \begin{itemize} \itemsep -1pt
% \item Wrote several Python and shell scripts which
%   automated recurring tasks -- such as checking network
%   interfaces on our routers, adding new user, blocking
%   specific ports on our internal routing network, etc.
%   The scripts themselves didn't use anything besides
%   standard library, the most important being asyncore
%   and asynchat.
% \end{itemize}


{\bf Part-time System Administrator,} Strum \hfill August 2005 -- February 2006

% Part time systems administrator at small research institution.

% \begin{itemize} \itemsep -1pt
% \item Administered small proxy server which was built on top of
%   commodity hardware and used Debian GNU/Linux distribution. Wrote
%   several scripts in combination of bash and awk for analyzing traffic usage
% \item Wrote small C++ program usign MFC that calculated impacts of different
%   types of explosions.
% \end{itemize}

\section{Skills}
\begin{tabular}{l p{3in}}
  Languages: & Python, Clojure, Emacs Lisp \\
  Libraries: & Flask, Werkzeug, Mapnik, Twisted, werkzeug, GeoAlchemy, lxml, Boost.Python \\
  Data persistence and messaging systems: & PostgreSQL, PostGIS, MongoDB, RabbitMQ, MySQL, Redis \\
  Tools: & GNU Emacs, coreutils, git
\end{tabular}

\section{Talks, Presentations and Code Sprints}

\begin{itemize}
\item July 2012 -- Talk ``Diving into Flask (Head On)'' at EuroPython 2012
\item June 2011 -- Talk ``Making use of OpenStreetMap data with Python'' at EuroPython 2011
\item March 2011 -- Tutorial ``Building your own tile server using OpenStreetMap'' at PyCon US 2011
\item September 2010 -- Mapnik Code Sprint \#1 in London
\item July 2010 -- Lightning talk on GIS-enabled databases at State of the Map 2010
\item January 2010 -- Talk ``Using Python at CloudMade'' at PyCamp Kyiv 2010
\item July 2009 -- Presentation of Vector Maps service at State of the Map 2009
\item May 2009 -- Improvised Mapnik Code Sprint at WhereCamp 2009 unconference
\end{itemize}

\section{Languages}
\begin{itemize} \itemsep -1pt
\item Russian, Ukrainian -- native
\item English -- fluent
\end{itemize}

\end{resume}

\end{document}
