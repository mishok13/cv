\documentclass[margin]{res}

\usepackage{helvetica}
\textwidth=5.2in
\usepackage{hyperref}
\hypersetup{colorlinks}
\usepackage{color}
\definecolor{darkred}{rgb}{0.4,0,0}
\definecolor{darkgreen}{rgb}{0,0.5,0}
\definecolor{darkblue}{rgb}{0,0,0.5}
\hypersetup{
  colorlinks,
  linkcolor=darkblue,
  filecolor=darkgreen,
  urlcolor=darkred,
  citecolor=darkblue}

\begin{document}

\name{Andrii V. Mishkovskyi\\[12pt]} % the \\[12pt] adds a blank line after name

\address{{\bf Mailing address} \\ Czaar Peterstraat 103B \\ 1018PE, Amsterdam, The Netherlands}

\address{{\bf Contacts} \\ +31 62 980 61 41 \\ contact@mishkovskyi.net \\ \href{https://github.com/mishok13}{mishok13@GitHub}}

\begin{resume}

\section{Summary}
Experienced developer with 7+ years of experience in various problem
domains, with deep knowledge of Python and Clojure, experience with
various RDBMS, NoSQL, distributed computing, messaging systems, cloud
computing platforms, etc. A quick learner, team player, open-source
aficionado. Always curious about the bleeding edge of technology and
computer science developments, while being a practical person when it
comes to choosing tools for work.

\section{Education}
Bachelor in Computer Science, National Technical University of Ukraine ``Kyiv Politechnical Institute'' \\
2003 --- 2007

\section{Experience}

{\bf Senior Data Engineer} TravelBird BV \hfill February 2015 --- present

TravelBird is a travel startup focusing on delivering exciting travel
packages every day. As part of BI team I'm taking part in helping the
company make educated, data-based choices as opposed to relying on gut
feeling.

{\bf Software Engineer/Team Lead} Screen6 \hfill March 2013 --- February 2015

Screen6 is a young startup attempting to provide advertisers with
cross-device services. As the first employee and subsequently a team
lead at Screen6 I am responsible for designing, coding and overlooking
most of the projects being developed at the company.\\

\begin{itemize} \itemsep -1pt
\item Led the design and architecture of real-time cross-device
  analytics dashboard, was involved in all parts of the system. The
  data pipeline resembles Lambda Architecture and is written in
  Clojure. Overlooked development of UI part of the dashboard,
  implemented in Python and CoffeeScript.
\item Led successful major refactoring of the system, coordinating
  effort of 3-member team on modularizing code and supporting internal
  Maven repository. Most of the code was Clojure with some libraries
  written in Java for convenience and performance reasons.
\item Built cookie de-duplication service for 3rd party advertising
  data providers. The system is capable of handling terabytes of
  advertising data daily and Screen6 has already successfully
  de-duplicated our clients' data. It's written in Clojure and node.js
  and is backed by Google's BigQuery service.
\item Overviewed design and development of real-time bidding service,
  capable of handling 25k req/s and more. Bidder was built in
  Coffeescript (on node.js) and Lua, with Redis as the storage.
\item Patches and bug reports to various open source projects ---
  Cascalog, Carmine, Storm, etc.
\end{itemize}

{\bf Core Developer} Hyves \hfill September 2011 --- February 2013

Hyves at the time was the biggest social network in Netherlands and
part of the biggest Dutch media corporation Telegraaf Media Groep. At
the time I've joined them Hyves' focus had shifted from building
social network functionality to end-users to providing services to 3rd
parties, including other parts of Telegraaf Media Groep.  I was part of
the team working on social identity platform, similar to Gigya,
Janrain and the like.\\

\begin{itemize} \itemsep -1pt
\item As part of the 4-man team, designed and developed social identity
  and API integration service, which is already in active use by several
  other projects. The service has been shown to be highly scalable and
  exhibits stable behavior even under real world stress conditions.
  The project is still in active development and the main focus right
  now is providing real-time statistics for 3rd parties.
\item As an official documentation lead of the aforementioned project,
  I have written and still maintain official documentation for both external
  users and the developers team themselves.
\item Was instrumental in moving the default technology stack choice
  from PHP and a set of proprietary libraries to Python and established
  open source libraries, such as Flask, SQLAlchemy, Jinja2, Celery and others.
  Take active part in teaching Python to other developers and system engineers.
\item Developed an internal web tool for editing proprietary i18n format.
  The tool is actively used by both engineering and marketing teams.
\item Active open-source involvement --- libredis, Flask-DebugToolBar, rauth, etc.
\end{itemize}

{\bf Senior Software Engineer,} CloudMade \hfill October 2009 --- August 2011\\
{\bf Software Engineer,} CloudMade \hfill August 2008 --- October 2009

CloudMade is GIS company that uses free (``free'' as in ``libre'')
data from OpenStreetMap project as the base for its services. I was
responsible for tile and vector servers development. Authorization
servers were also being supported by me.\\

\begin{itemize} \itemsep -1pt
\item Designed and implemented revamped version of the tile
  server. Tile server itself supports
  \href{http://developers.cloudmade.com/projects/show/tiles}{tile} and
  \href{http://developers.cloudmade.com/projects/show/static-maps}{staticmaps}
  services. Tile server was written in Twisted, enabling support for
  various networks protocols used and utilized RabbitMQ for deferring
  tough rendering tasks. Mapnik was used for actual rendering.  The
  previous version was an ad hoc mix of C, C++ and Python.
\item Designed and implemented cache management system, which is
  responsible for updating tiles and staticmaps images in the
  background, thus minimizing serve time and overall load on
  services. The software is written in Python and uses RabbitMQ (with
  another backend supporting BDB) for the update queue and SimpleDB
  (interchangable with PostgreSQL or BDB) for storing meta-information
  about each image re-rendered.
\item Designed and implemented vector maps service.
  \href{http://developers.cloudmade.com/projects/show/vector-stream-server}{Vectormaps}
  service is a very important part of
  navigation pack CloudMade provides and is used by many
  mobile applications.
\item Developed Python API
  \href{http://developers.cloudmade.com/projects/show/python-lib}{library}
  from scratch,
  which is  used by developers of Java and Ruby
  libraries as a model to follow.
\item Active open-source involvement --- Mapnik
  (PostGIS plugin, SRTM plugin,
  Python bindings), occasional patches to CPython.
\end{itemize}

{\bf Software Engineer,} Visonic Ltd. \hfill July 2007 --- August 2008

Visonic is an Israel-based manufacturer of security systems.
I was responsible for supporting legacy desktop software and
also got involved into development of several web-based
products.\\

\begin{itemize} \itemsep -1pt
\item Supported 10-year old MFC application, used for programming
  embedded devices. Written a Python application, using Qt to
  substitute the older software, but the project was not accepted
  because of management objections.
\item Wrote dozens Python and XSLT scripts
  crunching XML files. Most of these scripts had optional ncurses UI.
\item Created in-house wrapper of
  \href{http://www.hyperrealm.com/libconfig/}{libconfig} using SWIG for initial
  implementation and later rewriting it completely using only Python C API.
  The wrapping library was never released to opensource because of
  internal company decision.
\end{itemize}



{\bf Tech. Support Engineer,} Telesystems of Ukraine \hfill February 2006 --- July 2007

% Tier 1 support
% Worked as tech. support engineer for local ADSL internet
% service provider.\\

% \begin{itemize} \itemsep -1pt
% \item Wrote several Python and shell scripts which
%   automated recurring tasks --- such as checking network
%   interfaces on our routers, adding new user, blocking
%   specific ports on our internal routing network, etc.
%   The scripts themselves didn't use anything besides
%   standard library, the most important being asyncore
%   and asynchat.
% \end{itemize}


{\bf Part-time System Administrator,} Strum \hfill August 2005 --- February 2006

% Part time systems administrator at small research institution.

% \begin{itemize} \itemsep -1pt
% \item Administered small proxy server which was built on top of
%   commodity hardware and used Debian GNU/Linux distribution. Wrote
%   several scripts in combination of bash and awk for analyzing traffic usage
% \item Wrote small C++ program usign MFC that calculated impacts of different
%   types of explosions.
% \end{itemize}

\section{Skills}
\begin{tabular}{l p{3in}}
  Languages: & Python, Clojure, Emacs Lisp, Rust \\
  Libraries: & Cascalog, Guava, Flask, Werkzeug, Apache Commons, Cascading \\
  Persistence: & PostgreSQL, Redis, Google BigQuery, RabbitMQ, Hadoop \\
  Tools: & GNU Emacs, coreutils, git, Leiningen, fabric, awscli, puppet \\
\end{tabular}

\section{Talks, Presentations and Code Sprints}

Automated curated list can be seen on my \href{http://lanyrd.com/profile/mishok13/sessions/}{Lanyrd page}. The list below may be somewhat incomplete.

\begin{itemize}
\item July 2012 --- Talk ``Diving into Flask (Head On)'' at EuroPython 2012
\item June 2011 --- Talk ``Making use of OpenStreetMap data with Python'' at EuroPython 2011
\item March 2011 --- Tutorial ``Building your own tile server using OpenStreetMap'' at PyCon US 2011
\item September 2010 --- Mapnik Code Sprint \#1 in London
\item July 2010 --- Lightning talk on GIS-enabled databases at State of the Map 2010
\item January 2010 --- Talk ``Using Python at CloudMade'' at PyCamp Kyiv 2010
\item July 2009 --- Presentation of Vector Maps service at State of the Map 2009
\item May 2009 --- Improvised Mapnik Code Sprint at WhereCamp 2009 unconference
\end{itemize}

\section{Personal}

I'm a Ukrainian citizen, residing in The Netherlands under ``Kennismigrant'' (``knowledge migrant'') status. I'm not currently looking to relocating. I'm not open to remote work, freelance or part-time employment.

\end{resume}

\end{document}
